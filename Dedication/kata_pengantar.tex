% ******************************* Thesis Dedication ********************************

\begin{dedication} 

Dengan menyebut nama Allah Yang Maha Pengasih lagi Maha Penyayang, penulis menyelesaikan buku tugas akhir dengan judul:

\begin{center}
\textbf{Aerial Search and Traverse Robot (AESTRO):}
\textbf{Deteksi Korban Bencana Alam dengan Metode Deep Learning}
\end{center}

Tugas akhir ini merupakan salah satu syarat yang harus dipenuhi untuk meyelesaikan program studi Diploma 4 pada Jurusan Teknik Mekatronika Politeknik Elektronika Negeri Surabaya. Melalui kegiatan ini diharapkan mahasiswa dapat melakukan kegiatan laporan yang bersifat penelitian ilmiah dan menghubungkannya dengan teori yang telah diperoleh dalam perkuliahan. Buku ini juga disusun sepenuh hati dengan harapan pembaca mendapatkan ilmu dan gambaran dengan jelas tentang apa yang penulis kerjakan.

Pada kesempatan ini penulis panjatkan puji syukur kehadirat Allah SWT atas segala nikmat yang telah diberikan-Nya. Shalawat serta salam tidak lupa kita curahkan kepada junjungan nabi besar Nabi Muhammad SAW beserta keluarga, para sahabat dan umatnya hingga akhir zaman. Serta tidak lupa ucapan terima kasih yang sebesar-besarnya kepada beberapa pihak yang telah memberikan dukungan selama proses penyelesaian tugas akhir ini, antara lain:

\begin{enumerate}
    \item Orang tua penulis, Drs. Irwan Hermana dan Hamidah, serta seluruh keluarga besar yang selalu mengalirkan doa, memberikan semangat, nasehat, pengertian, dan dengan penuh kesabaran dalam membimbingku.
    \item Bapak Dr.Eng. Indra Adji Sulistijono, ST, M.Eng dan Anhar Risnumawan, S.ST., M.CS. selaku dosen pembimbing tugas akhir yang telah banyak membantu, membiayai dan membimbing hingga laporan ini dapat terselesaikan.
    \item Ibu Endah Suryawati Ningrum, S.T., M.T. selaku Kepala Program Studi Teknik Mekatronika PENS.
    \item Bapak dan Ibu dosen penguji tugas akhir yang telah memberikan saran dan masukannya kepada penulis.
    \item Semua Bapak dan Ibu dosen dilingkugan PENS khususnya Jurusan Teknik Mekatronika yang telah memberikan ilmu, nasehat, serta waktunya dengan ikhlas selama ini.
    \item Penghuni lab festo terutama temanku dalam tim AESTRO, Arif Dharmawan, yang berjuang bersama demi mimpinya masing masing.
    \item Teman-teman D4 Teknik Mekatronika 2013 yang luar biasa dan selalu saling menyemangati. Terutama untuk Bayu, Dimas dan Galuh atas referensi ilmu dan pengalamannya di wahana terbang.
    \item Teman-teman PENS, PENSSKY Venture, Surabaya.py dan TEDxTuguPahlawan yang telah memberi semangat, inspirasi dan kegiatan untuk melepas penat disela-sela pengerjaan tugas akhir.
    \item Serta semua pihak yang telah membantu kelancaran pelaksanaan tugas akhir yang tidak bias disebutkan satu persatu.
\end{enumerate} 

Dalam penyusunan laporan tugas akhir ini penulis menyadari akan adanya kekurangan-kekurangan baik dalam penyususnan maupun pembahasan masalah karena keterbatasan pengetahuan penulis. Untuk itu penulis mengharapkan kritik dan saran membangun dari seua pihak agar dapat lebih baik di masa yang akan datang. Terima kasih.

\end{dedication}

