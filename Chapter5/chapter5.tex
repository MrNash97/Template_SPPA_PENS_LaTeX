%!TEX root = ../thesis.tex
%*******************************************************************************
%****************************** Third Chapter **********************************
%*******************************************************************************
\chapter{PENUTUP}

% **************************** Define Graphics Path **************************
\ifpdf
    \graphicspath{{Chapter5/Figs/Raster/}{Chapter5/Figs/PDF/}{Chapter5/Figs/}}
\else
    \graphicspath{{Chapter5/Figs/Vector/}{Chapter5/Figs/}}
\fi

\section{Kesimpulan}
Dari hasil uji coba pada penelitian ini dapat ditarik beberapa kesimpulan:
\begin{enumerate}
\item Komunikasi wireless antara Xbee Pro dengan Notebook bisa dilakukan dengan menggunakan komunikasi USB – Serial. 
\item Dengan bantuan beberapa software, komunikasi serial secara wireless dapat diakses dengan mudah.
\item Struktur mekanik sangat berperan penting dalam mempermudah sistem kontrol pada robot, struktur yang kokoh dari sebuah robot sangat mempengaruhi tingkat kerumitan saat robot diprogram.
\item Robot iSRo dapat dikendalikan seraca manual dan wireless.
\end{enumerate}

\section{Saran}
Pada serangkaian uji coba dan penelitian yang dilakukan selama ini, robot iSRo dapat dikemudikan dengan menggunakan manual dan wireless. Dengan menggunakan bantuan beberapa software komunikasi antara user dengan robot bisa jauh lebih mudah. Semoga apa yang telah  tercapai pada saat ini dapat berguna bagi para pembaca. Segala kritik, saran dan masukan yang bersifat membangun sangat diharapkan untuk membantu kesempurnaan proyek ini nantinya.