% ************************** Thesis Abstract *****************************
% Use `abstract' as an option in the document class to print only the titlepage and the abstract.
\begin{abstract}
{\setstretch{1.2}{
%Kesehatan merupakan hal yang vital bagi setiap manusia. Melakukan pengecekan kesehatan secara berkala serta antisipasi penyakit sejak dini merupakan kebutuhan yang seharunya dilakukan oleh setiap orang supaya mendapatkan rasa aman karena terhindar dari penyakit serta mendapatkan penanganan yang cepat dan tepat saat terindikasi mengidap penyakit tertentu. Namun, hal tersebut sering diabaikan oleh kebanyakan orang. Hal itu dapat disebabkan oleh beberapa faktor, diantaranya adalah minimnya sarana kesehatan di suatu daerah, proses medical check-up yang memerlukan biaya yang tidak sedikit dan membutuhkan alat bantu yang perlu dipasangkan pada pasien yang memerlukan banyak waktu serta dapat menggangu kenyamanan pasien, sehingga mereka enggan pergi ke rumah sakit atau klinik untuk melakukan pemeriksaan kesehatan jika tidak mendapati gangguan kesehatan yang mengkhawatirkan  (Pusat Data dan Informasi Kementerian Kesehatan Republik Indonesia, 2017).

Kesehatan merupakan hal yang vital bagi setiap individu. Melakukan pengecekan kesehatan secara berkala serta antisipasi penyakit sejak dini merupakan kebutuhan yang seharunya dilakukan oleh setiap orang supaya mendapatkan rasa aman karena terhindar dari penyakit serta mendapatkan penanganan yang cepat dan tepat saat terindikasi mengidap penyakit tertentu. Berdasarkan penelitian \citet{fieselmann1993}, \textit{Breathing Rate} (BR) adalah salah satu indikator tanda vital utama, dan sering digunakan untuk menyimpulkan status kesehatan kardiopulmonari subjek. Sebagai contoh, tingkat pernapasan yang lebih tinggi dari 27 kali per menit adalah prediktor paling penting untuk pasien serangan jantung. Metode yang paling umum digunakan untuk mengukur HR dan BR adalah menghitung secara manual, selain itu HR dan BR dapat diukur menggunakan berbagai sensor yang dipasangkan pada tubuh. Namun, penggunaan sensor-sensor untuk monitor HR dan BR dalam praktisnya sulit untuk diterapkan dalam kondisi tertentu karena pasien cenderung merasa tidak nyaman, misalnya pada bayi yang aktif bergerak atau ketika gerakan bebas diperlukan, diagnosa luka (luka bakar / ulkus / trauma) dan evaluasi penyembuhan kulit. Oleh karena itu, penelitian ini bertujuan untuk merancang sebuah sistem yang dapat mengukur tingkat HR dan BR tanpa diperlukan kontak langsung dengan tubuh dengan menggunakan teknik \textit{Photoplethysmography} (PPG) yang memanfaatkan pengolahan citra dan pengolahan sinyal dengan  menggunakan kerangka kerja \textit{Eulerian Video Magnification (EMV)}.

{\noindent\setstretch{1.2}\textbf{Kata Kunci:} \textit{Kesehatan, Non contact-Monitoring, \textit{Photoplethysmography} (PPG), Pengolahan Citra, Pengolahan Sinyal, Eulerian Video Magnification (EVM).}}}

\addcontentsline{toc}{chapter}{ABSTRAK}
}\end{abstract}
